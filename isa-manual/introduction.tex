% -*- mode: latex; tab-width: 2; indent-tabs-mode: nil; -*-
%------------------------------------------------------------------------------
% MRISC32 ISA Manual - Introduction.
%
% This work is licensed under the Creative Commons Attribution-ShareAlike 4.0
% International License. To view a copy of this license, visit
% http://creativecommons.org/licenses/by-sa/4.0/ or send a letter to
% Creative Commons, PO Box 1866, Mountain View, CA 94042, USA.
%------------------------------------------------------------------------------

\chapter{Introduction}

\section{Overview}

MRISC32 is an open and free instruction set architecture (ISA).

It is a RISC style load-store vector architecture that is designed to be simple
yet highly scalable.

One of the main features of the instruction set architecture is its
forward-looking vector functionality that not only integrates well with the
rest of the ISA, but also enables implementations to freely select the level of
hardware parallelism. This makes the vector functionality suitable for low-end
systems like embedded microcontrollers as well as for high-performance
computing (HPC).

\todo{Add wording about goals.}

\section{Architecture modules}

The MRISC32 instruction set architecture consists of the mandatory Base
architecture, plus the following optional architecture modules:

\begin{bulletitems}
  \item Vector operation module (\hyperref[module:VM]{VM})
  \item Packed operation module (\hyperref[module:PM]{PM})
  \item Floating-point module (\hyperref[module:FM]{FM})
  \item Saturating and halving arithmetic module (\hyperref[module:SM]{SM})
\end{bulletitems}

\section{Data types}

\subsection{Size types}

The following types define a size without mandating any particular
interpretation of the data:

\begin{tabular}{|l|l|}
  \hline
  \textbf{Name} & \textbf{Size} \\
  \hline
  word & 32 bits \\
  \hline
  half-word & 16 bits \\
  \hline
  byte & 8 bits \\
  \hline
\end{tabular}

\subsection{Integer types}

Signed integer types are represented in two's complement form.

\begin{tabular}{|l|l|p{140pt}|}
  \hline
  \textbf{Name} & \textbf{Size} & \textbf{Meaning} \\
  \hline
  int32 & 32 & Signed 32-bit integer \\
  \hline
  uint32 & 32 & Unsigned 32-bit integer \\
  \hline
  int16 & 16 & Signed 16-bit integer \\
  \hline
  uint16 & 16 & Unsigned 16-bit integer \\
  \hline
  int8 & 8 & Signed 8-bit integer \\
  \hline
  uint8 & 8 & Unsigned 8-bit integer \\
  \hline
\end{tabular}

\subsection{Fixed point types}

For some operations the fixed point Q number format is used, in which the most
significant bit is the integer/sign bit, and the rest of the bits are the
fractional bits.

\subsubsection{Q31}

Q31 is a 32-bit signed fixed point number with 31 fractional bits.

The value of a Q31 number is obtained by interpreting the bit vector as a two's
complement signed integer multiplied by $2^{-31}$.

\begin{bytefield}{32}
  \bitheader{0,30,31} \\
  \bitbox{1}{I} &
  \bitbox{31}{F}
\end{bytefield}

\subsubsection{Q15}

Q15 is a 16-bit signed fixed point number with 15 fractional bits.

The value of a Q15 number is obtained by interpreting the bit vector as a two's
complement signed integer multiplied by $2^{-15}$.

\begin{bytefield}{16}
  \bitheader{0,14,15} \\
  \bitbox{1}{I} &
  \bitbox{15}{F}
\end{bytefield}

\subsubsection{Q7}

Q7 is an 8-bit signed fixed point number with 7 fractional bits.

The value of a Q7 number is obtained by interpreting the bit vector as a two's
complement signed integer multiplied by $2^{-7}$.

\begin{bytefield}{8}
  \bitheader{0,6,7} \\
  \bitbox{1}{I} &
  \bitbox{7}{F}
\end{bytefield}

\subsection{Floating-point types}

\begin{tabular}{|l|l|p{140pt}|}
  \hline
  \textbf{Name} & \textbf{Size} & \textbf{Meaning} \\
  \hline
  float32 & 32 & Single precision binary floating-point number \\
  \hline
  float16 & 16 & Half precision binary floating-point number \\
  \hline
  float8 & 8 & Quarter precision binary floating-point number \\
  \hline
\end{tabular}

\subsubsection{float32}

The float32 type uses one sign bit (S), eight exponent bits (E) and 23
fractional bits (F)\footnote{The float32 type uses the same format and interpretation as IEEE 754-2008 binary32}.

The significand has an implicit leading bit (to the left of the binary point)
with value 1, giving 24 effective significand bits.

The exponent bias is 127.

\begin{bytefield}{32}
  \bitheader{0,23,30,31} \\
  \bitbox{1}{S} &
  \bitbox{8}{E} &
  \bitbox{23}{F}
\end{bytefield}

\subsubsection{float16}

The float16 type uses one sign bit (S), five exponent bits (E) and ten
fractional bits (F)\footnote{The float16 type uses the same format and interpretation as IEEE 754-2008 binary16}.

The significand has an implicit leading bit (to the left of the binary point)
with value 1, giving eleven effective significand bits.

The exponent bias is 15.

\begin{bytefield}{16}
  \bitheader{0,10,14,15} \\
  \bitbox{1}{S} &
  \bitbox{5}{E} &
  \bitbox{10}{F}
\end{bytefield}

\subsubsection{float8}

The float8 type uses one sign bit (S), four exponent bits (E) and three
fractional bits (F).

The significand has an implicit leading bit (to the left of the binary point)
with value 1, giving four effective significand bits.

The exponent bias is 7.

\begin{bytefield}{8}
  \bitheader{0,3,6,7} \\
  \bitbox{1}{S} &
  \bitbox{4}{E} &
  \bitbox{3}{F}
\end{bytefield}

\section{Pseudocode syntax}

The operation that an instruction performs is described using pseudocode.

\subsection{Pseudocode scope}

The pseudocode for each instruction shall be regarded as a function that is
executed for each chunk of each element of the operation.

For a scalar operation, there is only a single element.

For a vector operation, the number of elements is dictated by the vector
operation.

The number of chunks and the size of each chunk is dictated by the packed
operation mode.

\begin{todobox}
Describe different types that are used in the pseudocode, such as
"bit vector", "integer", "unsigned integer" and "float"?

Come up with a better name than "chunk"?
\end{todobox}

\subsection{Notation}

The following notation is used in the pseudocode that describes the operation
of an instruction:

\begin{tabular}{lp{170pt}}
\textbf{Notation} & \textbf{Meaning} \\
x<$k$> & Bit $k$ of bit vector x \\
x<$k$:$l$> & Bits $k$ to $l$ of bit vector x \\
MEM[x,N] & $N$ consecutive bytes in memory starting at address $x$, interpreted as an $8\times N$-bit vector with little endian storage \\
a & 1st instruction operand \\
b & 2nd instruction operand \\
c & 3rd instruction operand \\
IM & IM field of the instruction word \\
T & T field of the instruction word \\
V & Vector mode (two bits) \\
bits & Chunk size, in bits \\
scale & Scale factor according to the T field (1 for format C instructions) \\
i & Vector element number \\
$\leftarrow$ & Assignment \\
+ & Addition \\
- & Subtraction \\
$\times$ & Multiplication \\
/ & Division \\
\% & Remainder of integer division \\
$=$ & Equal \\
$\neq$ & Not equal \\
$\leq$ & Less than or equal \\
$\geq$ & Greater than or equal \\
$\neg$ & Logical not \\
$\wedge$ & Logical or \\
$\vee$ & Logical and \\
\textasciitilde & Bitwise not \\
$|$ & Bitwise or \\
\& & Bitwise and \\
\textasciicircum & Bitwise xor \\
$>>$ & Zero-fill right-shift \\
$<<$ & Zero-fill left-shift \\
$>>_{s}$ & Sticky right-shift (fill with MSB) \\
$<<_{s}$ & Sticky left-shift (fill with LSB) \\
ones(N) & Bit vector of $N$ 1-bits \\
zeros(N) & Bit vector of $N$ 0-bits \\
int(x) & Interpret bit vector $x$ as a two's complement signed integer \\
uint(x) & Interpret bit vector $x$ as an unsigned integer \\
float(x) & Interpret bit vector $x$ as a floating-point number \\
log$_{2}$(x) & Binary logarithm of $x$ \\
max(x,y) & Maximum value of $x$ and $y$ \\
min(x,y) & Minimum value of $x$ and $y$ \\
sat(x,N) & Saturate integer $x$ to the range $[-2^{N-1},2^{N-1})$ \\
satu(x,N) & Saturate integer $x$ to the range $[0,2^{N})$ \\
isnan(x) & True if $x$ is not a number (NaN) \\
\end{tabular}

\section{Instruction encoding}

All instructions are encoded in 32 bits. There are five different encoding
formats, A, B, C, D and E, that mainly differ in the number and kinds of
instruction operands.

\begin{bytefield}{32}
  \bitheader{0,2,7,9,14,15,16,18,21,26,29,31} \\
  \begin{rightwordgroup}{A}
    \bitboxes*{1}{000000} &
    \bitbox{5}{Ra} &
    \bitbox{5}{Rb} &
    \bitbox{2}{V} &
    \bitbox{5}{Rc} &
    \bitbox{2}{T} &
    \bitbox{7}{OP}
  \end{rightwordgroup} \\
  \begin{rightwordgroup}{B}
    \bitboxes*{1}{000000} &
    \bitbox{5}{Ra} &
    \bitbox{5}{Rb} &
    \bitbox{1}{V} &
    \bitbox{6}{FN} &
    \bitbox{2}{T} &
    \bitboxes*{1}{11111} &
    \bitbox{2}{OP}
  \end{rightwordgroup} \\
  \begin{rightwordgroup}{C}
    \bitbox{6}{OP} &
    \bitbox{5}{Ra} &
    \bitbox{5}{Rb} &
    \bitbox{1}{V} &
    \bitbox{15}{IM}
  \end{rightwordgroup} \\
  \begin{rightwordgroup}{D}
    \bitboxes*{1}{110} &
    \bitbox{3}{OP} &
    \bitbox{5}{Ra} &
    \bitbox{21}{IM}
  \end{rightwordgroup} \\
  \begin{rightwordgroup}{E}
    \bitboxes*{1}{110111} &
    \bitbox{5}{Ra} &
    \bitbox{3}{OP}
    \bitbox{18}{IM}
  \end{rightwordgroup}
\end{bytefield}

\subsection{Instruction word fields}

The field names that are used in the instruction format descriptions are listed
in the table below:

\begin{tabular}{|l|l|}
  \hline
  \textbf{Name} & \textbf{Meaning} \\
  \hline
  OP & Operation \\
  \hline
  FN & Function (extended operation) \\
  \hline
  V  & Vector Mode \\
  \hline
  T  & Type \\
  \hline
  Ra & Destination/source register number (0-31) \\
  \hline
  Rb & Source register number (0-31) \\
  \hline
  Rc & Source register number (0-31) \\
  \hline
  IM & Immediate value \\
  \hline
\end{tabular}

Not all field types appear in all instruction formats.

The OP field in combination with the FN field (where applicable) is the main
identification of the instruction, and dictates what operation the instruction
shall perform.

The V field defines the scalar/vector configuration of the operands. The
scalar/vector operand configuration is a two-bit identifier. When only one bit
is provided by the V field, that bit is used as the most significant bit of the
identifier, and the least significant bit is implicitly zero.

Operand types (S for scalar, V for vector) for each operand positions relates
to the V identifier as follows (note that load/store instructions always
interpret the second operand - i.e. the base address - as a scalar):

\begin{tabular}{|l|l|l|}
  \hline
  \textbf{V} & \textbf{Default} & \textbf{Load/store} \\
  \hline
  00 & S,S[,S] & S,S,S \\
  \hline
  10 & V,V[,S] & V,S,S \\
  \hline
  11 & V,V,V & V,S,V \\
  \hline
  01 & V,V,fold(V) & V,S,fold(V) \\
  \hline
\end{tabular}

The register fields Ra, Rb and Rc refer to one scalar or vector register each,
according to the OP and V fields. For instance if a register operand refers to
a vector register, and the corresponding R-field has the value 21, then the
register operand is V21.

The first register operand, Ra, can be a source or a destination register
depending on the instruction, while Rb and Rc are always source registers.

The T field further defines the instruction. For most instructions it defines
the packed data type that is to be used. For load/store instructions it
defines a scaling factor for the register offset operand (i.e. the third
operand):

\begin{tabular}{|l|l|l|}
  \hline
  \textbf{T} & \textbf{Default} & \textbf{Load/store} \\
  \hline
  00 & One 32-bit word & *1 \\
  \hline
  01 & Four 8-bit bytes & *2 \\
  \hline
  10 & Two 16-bit half-words & *4 \\
  \hline
  11 & (reserved) & *8 \\
  \hline
\end{tabular}

The IM field provides an immediate value. The size of the IM field depends on
the instruction format, and the interpretation of the field further depends on
the OP field.

\subsection{Format A}

Format A instructions are used for instructions that require three register
operands, and support both vector and packed operations.

Format A can encode 124 major instructions (OP~$\in~[0,123]$).

\subsection{Format B}

Format B instructions are used for instructions that only require two register
operands (for instance unary operations). Both vector and packed operations are
supported.

Format B can encode 256 major instructions (OP~$\in~[0,3]$, FN~$\in~[0,63]$).

\subsection{Format C}

Format C instructions are used for instructions that require two register
operands and one immediate operand. Vector operations are supported (but not
packed operations).

In general each format C instruction has a corresponding format A encoding with
the same value of the OP field. For instance, the instruction ADD exists in
both format A and format C encodings.

Format C can encode 47 major instructions (OP~$\in~[1,47]$).

\subsection{Format D}

Format D is used for instructions that need to be able to express large
immediate values.

Format D can encode 7 major instructions (OP~$\in~[0,6]$).

\subsection{Format E}

Format E is used for conditional branch instructions.

Format E can encode 8 major instructions (OP~$\in~[0,7]$).

\subsection{Future extensions and encodings}

The following table lists the actual and maximum number of instructions per
instruction format:

\input{build/instruction-counts}

Encodings with the four most significant bits set to $1110$ or $1111$ are
reserved for future encoding formats (or for extending the number of possible
instructions for existing formats).

As can be seen, there is ample room for adding more instructions in future
versions of the ISA.

\section{Immediate value encoding}

The encoded width of an immediate operand depends on the instruction encoding
format, and for each instruction format there is one or more possible
interpretations (encodings) of the immediate operand (which encoding to use
depends on the instruction).

\begin{tabular}{|l|l|p{130pt}|}
  \hline
  \textbf{Format} & \textbf{Width} & \textbf{Immediate encodings} \\
  \hline
  C & 15 bits & \hyperref[imm:I15]{I15}, \hyperref[imm:I15HL]{I15HL} \\
  \hline
  D & 21 bits & \hyperref[imm:I21HL]{I21HL}, \hyperref[imm:I21H]{I21H}, \hyperref[imm:I21X4]{I21X4} \\
  \hline
  E & 18 bits & \hyperref[imm:I18X4]{I18X4} \\
  \hline
\end{tabular}

\subsection{I15}
\label{imm:I15}

The I15 format is encoded using 15 bits as follows:

\begin{bytefield}{15}
  \bitheader{0,14} \\
  \bitbox{1}{S}
  \bitbox{14}{Data}
\end{bytefield}

The immediate value is expanded into a 32-bit word as follows:

\begin{bytefield}{32}
  \bitheader{0,14,31} \\
  \bitboxes{1}{SSSSSSSSSSSSSSSSSS}
  \bitbox{14}{Data}
\end{bytefield}

\subsection{I15HL}
\label{imm:I15HL}

The I15HL format is encoded using 15 bits as follows:

\begin{bytefield}{15}
  \bitheader{0,1,13,14} \\
  \bitbox{1}{H} &
  \bitbox{1}{S} &
  \bitbox{12}{Data}
  \bitbox{1}{L} &
\end{bytefield}

When H=0, the immediate value is expanded into a 32-bit word as follows:

\begin{bytefield}{32}
  \bitheader{0,1,13,31} \\
  \bitboxes{1}{SSSSSSSSSSSSSSSSSSS}
  \bitbox{12}{Data}
  \bitbox{1}{L}
\end{bytefield}

When H=1, the immediate value is expanded into a 32-bit word as follows:

\begin{bytefield}{32}
  \bitheader{0,19,31} \\
  \bitbox{1}{S}
  \bitbox{12}{Data}
  \bitboxes{1}{LLLLLLLLLLLLLLLLLLL}
\end{bytefield}

\subsection{I21HL}
\label{imm:I21HL}

The I21HL format is encoded using 21 bits as follows:

\begin{bytefield}{21}
  \bitheader{0,1,19,20} \\
  \bitbox{1}{H} &
  \bitbox{1}{S} &
  \bitbox{18}{Data}
  \bitbox{1}{L} &
\end{bytefield}

When H=0, the immediate value is expanded into a 32-bit word as follows:

\begin{bytefield}{32}
  \bitheader{0,1,19,31} \\
  \bitboxes{1}{SSSSSSSSSSSSS}
  \bitbox{18}{Data}
  \bitbox{1}{L}
\end{bytefield}

When H=1, the immediate value is expanded into a 32-bit word as follows:

\begin{bytefield}{32}
  \bitheader{0,13,31} \\
  \bitbox{1}{S}
  \bitbox{18}{Data}
  \bitboxes{1}{LLLLLLLLLLLLL}
\end{bytefield}

\subsection{I21H}
\label{imm:I21H}

The I21H format is encoded using 21 bits as follows:

\begin{bytefield}{21}
  \bitheader{0,20} \\
  \bitbox{21}{Data}
\end{bytefield}

The immediate value is expanded into a 32-bit word as follows:

\begin{bytefield}{32}
  \bitheader{0,11,31} \\
  \bitbox{21}{Data}
  \bitboxes{1}{00000000000}
\end{bytefield}

\subsection{I21X4}
\label{imm:I21X4}

The I21X4 format is encoded using 21 bits as follows:

\begin{bytefield}{21}
  \bitheader{0,20} \\
  \bitbox{1}{S}
  \bitbox{20}{Data}
\end{bytefield}

The immediate value is expanded into a 32-bit word as follows:

\begin{bytefield}{32}
  \bitheader{0,2,22,31} \\
  \bitboxes{1}{SSSSSSSSSS}
  \bitbox{20}{Data}
  \bitboxes{1}{00}
\end{bytefield}

\subsection{I18X4}
\label{imm:I18X4}

The I18X4 format is encoded using 18 bits as follows:

\begin{bytefield}{18}
  \bitheader{0,17} \\
  \bitbox{1}{S}
  \bitbox{17}{Data}
\end{bytefield}

The immediate value is expanded into a 32-bit word as follows:

\begin{bytefield}{32}
  \bitheader{0,2,19,31} \\
  \bitboxes{1}{SSSSSSSSSSSSS}
  \bitbox{17}{Data}
  \bitboxes{1}{00}
\end{bytefield}

\section{Assembler syntax}

\tbd

