% -*- mode: latex; tab-width: 2; indent-tabs-mode: nil; -*-
%------------------------------------------------------------------------------
% MRISC32 ISA Manual - Introduction.
%
% This work is licensed under the Creative Commons Attribution-ShareAlike 4.0
% International License. To view a copy of this license, visit
% http://creativecommons.org/licenses/by-sa/4.0/ or send a letter to
% Creative Commons, PO Box 1866, Mountain View, CA 94042, USA.
%------------------------------------------------------------------------------

\chapter{Introduction}

\section{Overview}

MRISC32 is an open and free instruction set architecture (ISA).

\section{Notation}

\tbd

\section{Data types}

The following data types are recognized:

\begin{tabular}{|l|l|p{140pt}|}
  \hline
  \textbf{Name} & \textbf{Size} & \textbf{Description} \\
  \hline
  int8 & 8 & Signed 8-bit integer \\
  \hline
  uint8 & 8 & Unsigned 8-bit integer \\
  \hline
  int16 & 16 & Signed 16-bit integer \\
  \hline
  uint16 & 16 & Unsigned 16-bit integer \\
  \hline
  int32 & 32 & Signed 32-bit integer \\
  \hline
  uint32 & 32 & Unsigned 32-bit integer \\
  \hline
  Q7 & 8 & Signed fixed point number with 7 fractional bits \\
  \hline
  Q15 & 16 & Signed fixed point number with 15 fractional bits \\
  \hline
  Q31 & 32 & Signed fixed point number with 31 fractional bits \\
  \hline
  float8 & 8 & Quarter precision binary floating-point number \\
  \hline
  float16 & 16 & Half precision binary floating-point number \\
  \hline
  float32 & 32 & Single precision binary floating-point number \\
  \hline
\end{tabular}

\todo{Document each data type in more detail.}

\section{Assembler syntax}

\tbd

\section{Instruction formats}

All instructions are encoded in 32 bits. There are four different encoding
formats, A, B, C and D, that mainly differ in the number and kinds of
instruction operands.

The field names that are used in the instruction format descriptions are listed
in the table below:

\begin{tabular}{|l|l|}
  \hline
  \textbf{Name} & \textbf{Description} \\
  \hline
  OP & Operation \\
  \hline
  FN & Function (extended operation) \\
  \hline
  V  & Vector Mode \\
  \hline
  T  & Type \\
  \hline
  Ra & Destination/source register number (0-31) \\
  \hline
  Rb & Source register number (0-31) \\
  \hline
  Rc & Source register number (0-31) \\
  \hline
  H  & Immediate Hi/Lo flag \\
  \hline
  IM & Immediate value \\
  \hline
\end{tabular}

Not all field types appear in all instruction formats.

The OP field in combination with the FN field (where applicable) is the main
identification of the instruction, and dictates what operation the instruction
shall perform.

The V field defines the scalar/vector configuration of the operands. The
scalar/vector operand configuration is a two-bit identifier. When only one bit
is provided by the V field, that bit is used as the most significant bit of the
identifier, and the least significant bit is implicitly zero.

Operand types (S for scalar, V for vector) for each operand positions relates
to the V identifier as follows (note that load/store instructions always
interpret the second operand - i.e. the base address - as a scalar):

\begin{tabular}{|l|l|l|}
  \hline
  \textbf{V} & \textbf{Default} & \textbf{Load/store} \\
  \hline
  00 & S,S[,S] & S,S,S \\
  \hline
  10 & V,V[,S] & V,S,S \\
  \hline
  11 & V,V,V & V,S,V \\
  \hline
  01 & V,V,fold(V) & V,S,fold(V) \\
  \hline
\end{tabular}

The register fields Ra, Rb and Rc refer to one scalar or vector register each,
according to the OP and V fields. For instance if a register operand refers to
a vector register, and the corresponding R-field has the value 21, then the
register operand is V21.

The first register operand, Ra, can be a source or a destination register
depending on the instruction, while Rb and Rc are always source registers.

The T field further defines the instruction. For most instructions it defines
the packed data type that is to be used (for packed operations). For load/store
instructions it defines a scaling factor for the register offset operand (i.e.
the third operand):

\begin{tabular}{|l|l|l|}
  \hline
  \textbf{T} & \textbf{Default} & \textbf{Load/store} \\
  \hline
  00 & One 32-bit word & *1 \\
  \hline
  01 & Four 8-bit bytes & *2 \\
  \hline
  10 & Two 16-bit half-words & *4 \\
  \hline
  11 & (reserved) & *8 \\
  \hline
\end{tabular}

The IM field provides an immediate value. The size of the IM field depends on
the instruction format, and the interpretation of the field further depends on
the OP and the H fields.

The H field describes how to interpret a 14-bit IM field:
\begin{itemize}
  \item When H=0, the IM value is sign-extended to 32 bits.
  \item When H=1, the IM value is shifted to the left 18 position, and the
        least significant bit of the IM field is duplicated to the 18 least
        significant bits of the final immediate value.
\end{itemize}

\subsection{Format A}

\begin{bytefield}{32}
  \bitheader{0,7,9,14,16,21,26,31} \\
  \bitboxes*{1}{000000} &
  \bitbox{5}{Ra} &
  \bitbox{5}{Rb} &
  \bitbox{2}{V} &
  \bitbox{5}{Rc} &
  \bitbox{2}{T} &
  \bitbox{7}{OP}
\end{bytefield}

Format A instructions are used for instructions that require three register
operands, and support both vector and packed operations.

\subsection{Format B}

\begin{bytefield}{32}
  \bitheader{0,2,7,9,15,16,21,26,31} \\
  \bitboxes*{1}{000000} &
  \bitbox{5}{Ra} &
  \bitbox{5}{Rb} &
  \bitbox{1}{V} &
  \bitbox{6}{FN} &
  \bitbox{2}{T} &
  \bitboxes*{1}{11111} &
  \bitbox{2}{OP}
\end{bytefield}

Format B instructions are used for instructions that only require two register
operands (for instance unary operations). Both vector and packed operations are
supported.

\subsection{Format C}

\begin{bytefield}{32}
  \bitheader{0,14,15,16,21,26,31} \\
  \bitbox{6}{OP} &
  \bitbox{5}{Ra} &
  \bitbox{5}{Rb} &
  \bitbox{1}{V} &
  \bitbox{1}{H} &
  \bitbox{14}{IM}
\end{bytefield}

Format C instructions are used for instructions that require two register
operands and one immediate operand. Vector operations are supported (but not
packed operations).

In general each format C instruction has a corresponding format A encoding with
the same value of the OP field.

\subsection{Format D}

\begin{bytefield}{32}
  \bitheader{0,21,26,30,31} \\
  \bitboxes*{1}{11} &
  \bitbox{4}{OP} &
  \bitbox{5}{Ra} &
  \bitbox{21}{IM}
\end{bytefield}

Format D is used for instructions that need to be able to express large
immediate values.
