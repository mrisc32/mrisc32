% -*- mode: latex; tab-width: 2; indent-tabs-mode: nil; -*-
%------------------------------------------------------------------------------
% MRISC32 ISA Manual - Introduction.
%
% This work is licensed under the Creative Commons Attribution-ShareAlike 4.0
% International License. To view a copy of this license, visit
% http://creativecommons.org/licenses/by-sa/4.0/ or send a letter to
% Creative Commons, PO Box 1866, Mountain View, CA 94042, USA.
%------------------------------------------------------------------------------

\chapter{Introduction}

\section{Overview}

MRISC32 is an open and free instruction set architecture (ISA).

The MRISC32 ISA supports a subset of the 2008 IEEE-754 floating-point
standard~\cite{ieee754-2008}.

\section{Notation}

\tbd

\section{Assembler syntax}

\tbd

\section{Instruction formats}

All instructions are encoded in 32 bits. There are four different encoding
formats, A, B, C and D, that mainly differ in the number and kinds of
instruction operands.

The field names that are used in the instruction format descriptions are listed
in the table below:

\begin{tabular}{|l|l|}
  \hline
  \textbf{Name} & \textbf{Description} \\
  \hline
  OP & Operation \\
  \hline
  FN & Function (extended operation) \\
  \hline
  R1 & Destination/source register number (0-31) \\
  \hline
  R2 & Source register number (0-31) \\
  \hline
  R3 & Source register number (0-31) \\
  \hline
  IM & Immediate value \\
  \hline
  V  & Vector Mode \\
  \hline
  T  & Type \\
  \hline
  H  & Immediate Hi/Lo flag \\
  \hline
\end{tabular}

\todo{Describe the fields in more detail.}

\subsection{Format A}

\begin{bytefield}{32}
  \bitheader{0,7,9,14,16,21,26,31} \\
  \bitboxes*{1}{000000} &
  \bitbox{5}{R1} &
  \bitbox{5}{R2} &
  \bitbox{2}{V} &
  \bitbox{5}{R3} &
  \bitbox{2}{T} &
  \bitbox{7}{OP}
\end{bytefield}

\todo{Describe the general use of format A instructions.}

\subsection{Format B}

\begin{bytefield}{32}
  \bitheader{0,2,7,9,15,16,21,26,31} \\
  \bitboxes*{1}{000000} &
  \bitbox{5}{R1} &
  \bitbox{5}{R2} &
  \bitbox{1}{V} &
  \bitbox{6}{FN} &
  \bitbox{2}{T} &
  \bitboxes*{1}{11111} &
  \bitbox{2}{OP}
\end{bytefield}

\todo{Describe the general use of format B instructions.}

\subsection{Format C}

\begin{bytefield}{32}
  \bitheader{0,14,15,16,21,26,31} \\
  \bitbox{6}{OP} &
  \bitbox{5}{R1} &
  \bitbox{5}{R2} &
  \bitbox{1}{V} &
  \bitbox{1}{H} &
  \bitbox{14}{IM}
\end{bytefield}

\todo{Describe the general use of format C instructions.}

\subsection{Format D}

\begin{bytefield}{32}
  \bitheader{0,21,26,30,31} \\
  \bitboxes*{1}{11} &
  \bitbox{4}{OP} &
  \bitbox{5}{R1} &
  \bitbox{21}{IM}
\end{bytefield}

\todo{Describe the general use of format D instructions.}
