% -*- mode: latex; tab-width: 2; indent-tabs-mode: nil; -*-
%------------------------------------------------------------------------------
% MRISC32 ISA Manual - Application Binary Interface.
%
% This work is licensed under the Creative Commons Attribution-ShareAlike 4.0
% International License. To view a copy of this license, visit
% http://creativecommons.org/licenses/by-sa/4.0/ or send a letter to
% Creative Commons, PO Box 1866, Mountain View, CA 94042, USA.
%------------------------------------------------------------------------------

\chapter{Application Binary Interface}
\label{sec:abi}

This chapter contains recommendations for platform application binary
interfaces (ABIs). It is not a complete ABI specification.

\section{Calling convention}

\subsection{Scalar registers}

\begin{tabular}{|c|c|l|}
  \hline
  \textbf{Register} & \textbf{Alias} & \textbf{Role and rule} \\
  \hline
  Z & R0 & Always zero (read only) \\
  \hline
  R1-R8 & & Function arguments / results \\
  \hline
  R9-R14 & & Temporary registers \\
  \hline
  R15 & & Intra-procedure call scratch register / temporary register \\
  \hline
  R16-R26 & & Callee-saved registers \\
  \hline
  TP & R27 & Thread pointer (callee-saved) \\
  \hline
  FP & R28 & Frame pointer (callee-saved) \\
  \hline
  SP & R29 & Stack pointer (callee-saved) \\
  \hline
  LR & R30 & Link register (callee-saved) \\
  \hline
  VL & R31 & Vector length (callee-saved) \\
  \hline
\end{tabular}

\subsubsection{Function arguments / results}

The first arguments to a function are passed in registers R1 to R8. How many
registers are used depends on the number of arguments and the types of the
arguments. For more information, see \ref{sec:abi_function_arguments}.

Likewise function results are returned in R1 to R8. For more information, see
\ref{sec:abi_function_results}.

These registers may also be used as temporary registers.

\subsubsection{Temporary registers}

Temporary registers are not guaranteed to be preserved across function call
boundaries, and thus need not be preserved by the callee.

\subsubsection{Callee-saved registers}

The contents of callee-saved registers must be preserved by a function. This is
normally done by the function prologue and epilogue by storing and restoring
the registers to and from the stack.

\subsubsection{Intra-procedure call scratch register}

The intra-procedure call scratch register may be used for call target address
calculations. It may also be used as a temporary register.

\subsubsection{Thread pointer}

The thread pointer may be used by systems that need to provide fast access to
thread local data. Otherwise it may be used as a general purpose register.

The thread pointer is a callee-saved register.

\subsubsection{Frame pointer}

\tbd

\subsubsection{Stack pointer}

Upon function entry, the stack pointer contains the address of the top of the
stack. For more information, see \ref{sec:abi_stack}.

The stack pointer is a callee-saved register.

\subsubsection{Link register}

The link register contains the return address to the caller.

The link register is a callee-saved register.

\subsubsection{Vector length}

The vector length is a callee-saved register.

\subsection{Vector registers}

\begin{tabular}{|c|c|l|}
  \hline
  \textbf{Register} & \textbf{Alias} & \textbf{Role and rule} \\
  \hline
  VZ & V0 & Always zero (read only) \\
  \hline
  V1-V8 & & Function arguments / results \\
  \hline
  V9-V31 & & Temporary registers \\
  \hline
\end{tabular}

\subsubsection{Function arguments / results}

The first vector arguments to a function are passed in registers V1 to V8. How
many registers are used depends on the number of arguments.

Likewise function vector results are returned in V1 to V8.

These registers may also be used as temporary registers.

\subsubsection{Temporary registers}

All vector registers are temporary registers, and thus need not be preserved
by the callee.

\subsection{Stack}
\label{sec:abi_stack}

\tbd

\subsection{Function arguments}
\label{sec:abi_function_arguments}

\tbd

\subsection{Function results}
\label{sec:abi_function_results}

\tbd

\section{Data organization}

\subsection{Endianness}

Data fields are stored in memory using little endian representation. Thus the
least significant byte of a data field is at the lowest byte address that the
data field occupies in memory.

\subsection{Alignment}

Data fields that are one, two or four bytes in size shall be aligned to a
memory address that is divisable by the data field size.

Data fields that are larger than four bytes in size shall be aligned to a
memory address that is divisable by four.

\begin{tabular}{|l|l|l|}
  \hline
  \textbf{Type} & \textbf{Size (bytes)} & \textbf{Alignment (bytes)} \\
  \hline
  byte & 1 & 1 \\
  \hline
  half-word & 2 & 2 \\
  \hline
  word & 4 & 4 \\
  \hline
  double-word & 8 & 4 \\
  \hline
  quad-word & 16 & 4 \\
  \hline
\end{tabular}
