% -*- mode: latex; tab-width: 2; indent-tabs-mode: nil; -*-
%------------------------------------------------------------------------------
% MRISC32 ISA Manual - Instructions.
%
% This work is licensed under the Creative Commons Attribution-ShareAlike 4.0
% International License. To view a copy of this license, visit
% http://creativecommons.org/licenses/by-sa/4.0/ or send a letter to
% Creative Commons, PO Box 1866, Mountain View, CA 94042, USA.
%------------------------------------------------------------------------------

\chapter{Instructions}

This chapter describes all the instructions of the MRISC32 instruction set.

Instruction variants with a .B (packed byte) or .H (packed half-word) mnemonic
suffix are only available in implementations that support the Packed operation
module (\hyperref[module:PM]{PM}).

Instruction variants that include vector register operands are only available
in implementations that support the Vector operation module
(\hyperref[module:VM]{VM}).

For instructions that are not part of the Base architecture, the required
architecture module (or modules) is indicated in the instruction documentation.

The encoding format used for immediate operands is documented per instruction
(the IM field, if any, references the immediate encoding format).

\section{Pseudocode}

The operation that an instruction performs is described using pseudocode.

\subsection{Pseudocode scope}

The pseudocode for each instruction shall be regarded as a function that is
executed for each chunk of each element of the operation.

For a scalar operation, there is only a single element.

For a vector operation, the number of elements is dictated by the vector
operation.

The number of chunks and the size of each chunk is dictated by the packed
operation mode.

\todo{Add a diagram that visualizes how chunks are processed.}

\subsection{Types}

\subsubsection{bit vector}

A vector of bits of a given size, without any particular interpretation of the
meaning of the bits.

Instruction source and destination operands are treated as bit vectors. To
perform arithmetic operations, a bit vector must first be interpreted as an
integer or real value.

Example of an 8-bit bit vector: $00101101_2$.

\subsubsection{integer}

An integer value in the range $(-\infty, +\infty)$.

Integers support integer arithmetic operations.

Example: $-12345$.

\subsubsection{real}

A real value in the range $(-\infty, +\infty)$, with infinite precision.

Real values support real arithmetic operations.

Example: $-123.45$.

\subsection{Type conversions}

Type conversions can either be explicit or implicit.

Explicit conversions are typically used for interpreting a bit vector as an
integer or real value, e.g. in order to perform arithmetic opertaions. This can
be done with pseudocode functions such as uint($x$) and float($x$).

Implicit conversions are used when interpreting an integer or real value as a
bit vector, e.g. for assignment of the destination operand (which is always a
bit vector) or when performing bitwise or shift operations on an integer value.

An implicit conversion to a bit vector is done as follows:

\begin{bulletitems}
  \item Integer values are converted to a two's complement form bit vector of
        infinite width, which is then truncated to the target width.
  \item Real values are converted to an IEEE 754 binary bit vector
        representation of the target width.
\end{bulletitems}

\subsection{Numeric constants}

Unless otherwise noted, numeric constants are given as decimal (base 10)
integers.

Integers in other bases are given as $N_{base}$ (e.g. $101_{2}$).

Real values are given in base 10 (e.g. $10.2$).

\subsection{Notation}

The following notation is used in the pseudocode that describes the operation
of an instruction:

\begin{tabular}{lp{340pt}}
\textbf{Notation} & \textbf{Meaning} \\
REGa, REGb, REGc & Register number fields of the instruction word \\
IM & IM field of the instruction word \\
T & T field of the instruction word \\
V & Vector mode (two bits) \\
a, b, c & 1st, 2nd and 3rd operation operand (chunk bit vectors) \\
bits & Chunk size, in bits \\
scale & Scale factor according to the T field (1 for format C instructions) \\
i & Vector element number \\
$x$<$k$> & Bit $k$ of bit vector $x$ \\
$x$<$k$:$l$> & Bits $k$ to $l$ of bit vector $x$ \\
MEM[$x$,$N$] & $N$ consecutive bytes in memory starting at address $x$,
               interpreted as an $8\times N$-bit vector with little endian
               storage \\
SR[$x$] & System register number $x$ \\
$\leftarrow$ & Assignment \\
+, - & Addition, Subtraction \\
$\times$, / & Multiplication, Division \\
\% & Remainder of integer division \\
$=$, $\neq$ & Equal, Not equal \\
$<$, $>$ & Less than, Greater than \\
$\leq$, $\geq$ & Less than or equal, Greater than or equal \\
$\neg$, $\vee$, $\wedge$ & Logical NOT, OR, AND \\
\textasciitilde, $|$, \&, \textasciicircum & Bitwise NOT, OR, AND, XOR \\
$<<$, $>>$ & Zero-fill left-shift, right-shift \\
$<<_{s}$, $>>_{s}$ & Sticky left-shift (fill with LSB), right-shift (fill with MSB) \\
ones($N$) & Bit vector of $N$ 1-bits \\
zeros($N$) & Bit vector of $N$ 0-bits \\
int($x$) & Interpret bit vector $x$ as a two's complement signed integer.
           Returns an integer. \\
uint($x$) & Interpret bit vector $x$ as an unsigned integer. Returns an
            integer. \\
float($x$) & Interpret bit vector $x$ as a floating-point number. Returns a real
             value. \\
max($x$,$y$) & Maximum value of $x$ and $y$ \\
min($x$,$y$) & Minimum value of $x$ and $y$ \\
sat($x$,$N$) & Saturate integer $x$ to the range $[-2^{N-1},2^{N-1})$ \\
satu($x$,$N$) & Saturate integer $x$ to the range $[0,2^{N})$ \\
isnan($x$) & True if bit vector $x$ represents an IEEE 754 NaN value (not a
             number) \\
int2real($x$) & Convert integer value to a real value \\
trunc($x$) & Convert real value to an integer value, rounding towards zero (i.e.
             truncate) \\
round($x$) & Convert real value to an integer value, rounding towards the
             nearest value (halfway cases are rounded away from zero) \\
pow($x$, $y$) & Compute the value of $x$ raised to the power $y$, i.e. $x^{y}$
\end{tabular}

\clearpage

\input{build/gen-instructions}
