% -*- mode: latex; tab-width: 2; indent-tabs-mode: nil; -*-
%------------------------------------------------------------------------------
% MRISC32 ISA Manual - Examples.
%
% This work is licensed under the Creative Commons Attribution-ShareAlike 4.0
% International License. To view a copy of this license, visit
% http://creativecommons.org/licenses/by-sa/4.0/ or send a letter to
% Creative Commons, PO Box 1866, Mountain View, CA 94042, USA.
%------------------------------------------------------------------------------

% We do assembler examples in single-column mode to better fit the code.
\onecolumn

\chapter{Examples}

This is a non-normative section that contains programs that exemplify various
aspects of the MRISC32 instruction set architecture.

\section{Vector operation}

\subsection{saxpy}

\begin{lstlisting}[style=assembler]
; void saxpy(size_t n, const float a, const float *x, float *y)
; {
;   for (size_t i = 0; i < n; i++)
;     y[i] = a * x[i] + y[i];
; }
;
; Register arguments:
;   s1 - n
;   s2 - a
;   s3 - x
;   s4 - y

saxpy:
    bz    s1, 2f        ; Nothing to do?
    cpuid s5, z, z      ; Query the maximum vector length
1:
    minu  vl, s5, s1    ; Define the operation vector length
    sub   s1, s1, vl    ; Decrement loop counter
    ldw   v1, s3, #4    ; Load x (element stride = 4 bytes)
    ldw   v2, s4, #4    ; Load y
    fmul  v1, v1, s2    ; x * a
    fadd  v1, v1, v2    ; + y
    stw   v1, s4, #4    ; Store y
    ldea  s3, s3, vl*4  ; Increment array pointers
    ldea  s4, s4, vl*4  ; Increment array pointers
    bnz   s1, 1b
2:
    ret
\end{lstlisting}

\twocolumn
