% -*- mode: latex; tab-width: 2; indent-tabs-mode: nil; -*-
%------------------------------------------------------------------------------
% MRISC32 ISA Manual - Vector operation module.
%
% This work is licensed under the Creative Commons Attribution-ShareAlike 4.0
% International License. To view a copy of this license, visit
% http://creativecommons.org/licenses/by-sa/4.0/ or send a letter to
% Creative Commons, PO Box 1866, Mountain View, CA 94042, USA.
%------------------------------------------------------------------------------

\chapter{Vector operation module (VM)}
\label{module:VM}

The Vector operation module adds facilities for vector processing. A set of
vector registers is added, and most instructions are extended to support
processing of vector registers.

\section{Vector registers}

There are 32 vector registers:

\begin{tabular}{|l|l|}
  \hline
  \textbf{Vector reg. no} & \textbf{Name} \\
  \hline
  0 & VZ \\
  \hline
  1 & V1 \\
  \hline
  2 & V2 \\
  \hline
  3 & V3 \\
  \hline
  4 & V4 \\
  \hline
  \multicolumn{2}{c}{$\vdots$} \\
  \hline
  30 & V30 \\
  \hline
  31 & V31 \\
  \hline
\end{tabular}

Each register, V$k$, consists of $N$ 32-bit elements, where $N$ is
implementation defined ($N$ must be a power of two, and at least 16):

\begin{bytefield}{32}
  \bitheader{0,31} \\
  \wordbox{1}{V$k[0]$} \\
  \wordbox{1}{V$k[1]$} \\
  \wordbox{1}{V$k[2]$} \\
  \wordbox{1}{V$k[3]$} \\
  \wordbox{1}{V$k[4]$} \\
  \wordbox[]{1}{$\vdots$} \\[1ex]
  \wordbox{1}{V$k[N-2]$} \\
  \wordbox{1}{V$k[N-1]$}
\end{bytefield}

\subsection{The VZ register}

VZ is a read-only register with all vector elements set to zero. Writing to the
VZ register has no effect.

\section{Vector operation}

A vector operation is performed when a source or destination operand of an
instruction is a vector register.

\subsection{Vector length}

The vector length is the number of vector elements to process in a vector
operation.

All vector operations use the vector length that is given by the value of the
VL register at the time of instruction invocation.

When the vector length is $M$, vector elements $[0, M)$ are processed.

To obtain the maximum vector length for the implementation, read the
\hyperref[reg:MAX_VL]{MAX\_VL} system register.

\begin{notebox}
  The maximum vector length, as advertised by the MAX\_VL system register,
  reflects the implementation dependent vector register size. By respecting the
  value of MAX\_VL, software can be executed on implementations with different
  vector register sizes without modification.
\end{notebox}

\begin{todobox}
  The vector length should be defined by the TBD vector register length (per
  vector register tag).
\end{todobox}

\subsection{Folding}

Horizontal vector operations (e.g. sum and min/max) are supported by repeated
folding, where the upper half of one vector source operand is combined with the
lower half of another vector source operand.

\begin{todobox}
  Describe how folding works.
\end{todobox}

\subsection{Masking}

\begin{todobox}
  Define and describe masked vector operations.
\end{todobox}

\subsection{Operation}

A vector operation is performed as if all vector elements are processed as a
series of scalar operations, in order from the lowest vector element index to
the highest vector element index of the operation.

\begin{notebox}
  An implementation may process several vector elements concurrently in order
  to increase the operation throughput, but it is not a requirement.
\end{notebox}

The following sections describe how a vector operation is executed for different
operand configurations. In each description the following applies:

\begin{bulletitems}
  \item \texttt{VL} is the vector length of the operation
  \item \texttt{operation} is the operation to perform, as described by the instruction
  \item \texttt{Va, Vb, Vc} are vector register operands
  \item \texttt{Rb, Rc} are scalar register operands
  \item \texttt{IMM} is a scalar immediate operand
  \item \texttt{scale} is the optional index scale operand for load/store (1, 2, 4 or 8)
\end{bulletitems}

\subsubsection{Vector, Vector, Vector}

\begin{lstlisting}[style=pseudocode]
for i in 0 to VL-1 do
  operation(Va[i], Vb[i], Vc[i])
\end{lstlisting}

\subsubsection{Vector, Vector, Scalar register}

\begin{lstlisting}[style=pseudocode]
for i in 0 to VL-1 do
  operation(Va[i], Vb[i], Rc)
\end{lstlisting}

\subsubsection{Vector, Vector, Scalar immediate}

\begin{lstlisting}[style=pseudocode]
for i in 0 to VL-1 do
  operation(Va[i], Vb[i], #IMM)
\end{lstlisting}

\subsubsection{Vector, Scalar, Scalar register (load/store)}

\begin{lstlisting}[style=pseudocode]
for i in 0 to VL-1 do
  operation(Va[i], Rb, i * Rc * scale)
\end{lstlisting}

\subsubsection{Vector, Scalar, Scalar immediate (load/store)}

\begin{lstlisting}[style=pseudocode]
for i in 0 to VL-1 do
  operation(Va[i], Rb, i * IMM)
\end{lstlisting}

\subsubsection{Vector, Vector, Vector - Folding}

\begin{lstlisting}[style=pseudocode]
for i in 0 to VL-1 do
  operation(Va[i], Vb[VL+i], Vc[i])
\end{lstlisting}
