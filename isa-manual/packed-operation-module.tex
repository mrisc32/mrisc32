% -*- mode: latex; tab-width: 2; indent-tabs-mode: nil; -*-
%------------------------------------------------------------------------------
% MRISC32 ISA Manual - Packed operation module.
%
% This work is licensed under the Creative Commons Attribution-ShareAlike 4.0
% International License. To view a copy of this license, visit
% http://creativecommons.org/licenses/by-sa/4.0/ or send a letter to
% Creative Commons, PO Box 1866, Mountain View, CA 94042, USA.
%------------------------------------------------------------------------------

\chapter{Packed operation module}

The packed operation module adds facilities for parallel operation on packed
data types. Most instructions are extended with packed operation modes, and a
few instructions are added that mainly deal with packing and unpacking of data
of different sizes.

\section{Packed data operation}

Many instructions are extended with the ability to operate on several
individual sub-parts of the source and destination elements. These sub-parts
are referred to as chunks.

A single 32-bit element may be split up into one, two or four chunks, as
follows:

\begin{bytefield}{32}
  \bitheader{0,8,16,24,31} \\
  \begin{rightwordgroup}{}
    \bitbox{32}{word}
  \end{rightwordgroup} \\
  \begin{rightwordgroup}{H}
    \bitbox{16}{half-word} &
   \bitbox{16}{half-word}
  \end{rightwordgroup} \\
  \begin{rightwordgroup}{B}
    \bitbox{8}{byte} &
    \bitbox{8}{byte} &
    \bitbox{8}{byte} &
    \bitbox{8}{byte}
  \end{rightwordgroup}
\end{bytefield}

When a packed operation is performed, all chunks within a 32-bit word are
processed in parallel. It is not possible to process only a subset of the
chunks.

\subsection{Word mode}

In word mode, which is the defult, each element is processed as a single
32-bit chunk.

\subsection{Half-word mode}

In half-word mode each element is processed as two individual 16-bit chunks in
parallel.

In assembly language, half-word mode is indicated by appending the suffix .H
to the instruction mnemonic.

\subsection{Byte mode}

In byte mode each element is processed as four individual 8-bit chunks in
parallel.

In assembly language, byte mode is indicated by appending the suffix .B to the
instruction mnemonic.

\subsection{Packed floating-point operation}

For floating-point instructions, using packed operating modes implies using
floating-point precisions lower than single precision:

\begin{tabular}{|l|l|}
  \hline
  \textbf{Mode} & \textbf{Precision} \\
  \hline
  word & Single precision floating-point \\
  \hline
  half-word & Half precision floating-point \\
  \hline
  byte & Quarter precision floating-point \\
  \hline
\end{tabular}
